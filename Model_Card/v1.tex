\documentclass{article}
\usepackage[utf8]{inputenc}

% Used in the explanation text
\usepackage{hyperref}
\hypersetup{
    colorlinks = true,
    citecolor = {blue},
    urlcolor = {blue},
}

% Used by the template
\usepackage{setspace}
\usepackage{changepage} % to adjust margins
\usepackage[breakable]{tcolorbox}
\usepackage{float} % for tables inside tcolorbox https://tex.stackexchange.com/a/274342
\usepackage{enumitem}
\usepackage{geometry}

\geometry{vmargin=30pt,hmargin=30pt}
\begin{document}


% Each section is supposed to be brief, in the form of a bullet list.
% This environment formats the lists in each model card section in a compact format to help
% the card fit into the recommended "one to two pages".
\newenvironment{mcsection}[1]
    {%
        \textbf{#1}

        % Reduce margins to use the space more effectively and help fit in the recommended "one to two pages"
        % Use the bullet list format as shown in the model card paper to increase readability
        \begin{itemize}[leftmargin=*,topsep=0pt,itemsep=-1ex,partopsep=1ex,parsep=1ex,after=\vspace{\medskipamount}]
    }
    {%
        \end{itemize}
    }

% Optional: reduce margins single line to fit in "one to two pages", as recommended
\begin{singlespace}

\tcbset{colback=white!10!white}
\begin{tcolorbox}[title=\textbf{Model Card },
    breakable, sharp corners, boxrule=0.7pt]

% Change to a smaller, but still legible font size to help fit in the recommended "one to two pages"
\small{


\begin{mcsection}{Detalhes do Modelo}
    \item Esse modelo (versão 1.0) foi desenvolvido por Luís Eduardo Limas Brito, 27/01/2005
    \item Ele foi criado a fim de estabelecer uma base para aplicação de metodologias de IA responsável. Por isso, ele não contém nenhuma implementação que siga as dimensões estabelecidas pela IAR.
    \item Modelo de Predição, com o objetivo de estimar o número de internações por doenças respiratórias (CID = J...) em um hospital para um dado mês.
\end{mcsection}

\begin{mcsection}{Uso pretendido}
    \item O modelo pode ser usado por hospitais para analisar estimativas de quantas internações podem-se esperar para o próximo mês.
\end{mcsection}

\begin{mcsection}{Fatores}
    \item Factors 1...
\end{mcsection}

\begin{mcsection}{Metrics}
    \item Metrics 1....
\end{mcsection}

\begin{mcsection}{Evaluation Data}
    \item Evaluation data 1...
\end{mcsection}

\begin{mcsection}{Training Data}
    \item Training data 1...
\end{mcsection}

\pagebreak

\begin{mcsection}{Ethical Considerations}
    \item Ethical considerations 1....
\end{mcsection}

\begin{mcsection}{Caveats and Recommendations}
    \item Caveats and recommendations 1...
\end{mcsection}

\textbf{Quantitative Analyses}

% Sample table inside tcolorbox
\begin{table}[H]
\centering
\small{
\begin{tabular}{lr}
Measurement 1       & 0.751  \\
Measurement 2       & 0.762 \\
Measurement 3       & 0.773 \\
Measurement 4       & 0.784 \\
Measurement average & 0.768  \\ \hline
\textbf{Model measurement}  & \textbf{0.791} \\ \hline
\end{tabular} } \\
\caption[Short caption used in list of tables.]{\small{Longer caption to explain what the measurements are.}}
\end{table}

} % end font size change
\end{tcolorbox}
\end{singlespace}

\end{document}
